%\documentclass[twocolumn,floatfix,aps,showpacs,showkeys]{revtex4-1}
\documentclass[preprint,floatfix,aps,showpacs,showkeys]{revtex4-1}
%\documentclass[a4paper,12pt,fleqn]{article}
\usepackage{epsfig,subfigure,latexsym}
%\usepackage{rotating}
%------------------------------------
\usepackage{color}
%\usepackage{graphicx}
\usepackage{amssymb,amsmath}
\usepackage{epstopdf}
%\usepackage{bm}
\newcommand {\bea}{\begin{eqnarray}}
\newcommand {\eea}{\end{eqnarray}}
\newcommand {\be}{\begin{equation}}
\newcommand {\ee}{\end{equation}}
\newcommand {\pslash}{p\!\!\!/}

\begin{document}
\def\Journal#1#2#3#4{{\it #1} {\bf #2}, #3 (#4)}
\def\RPP{{Rep. Prog. Phys}}
\def\PRC{{Phys. Rev. C}}
\def\PRD{{Phys. Rev. D}}
\def\PRB{{Phys. Rev. B}}
\def\PRA{{Phys. Rev. A}}
\def\ZPA{{Z. Phys. A}}
\def\NPA{{Nucl. Phys. A}} 
\def\JPG{{J. Phys. G }}
\def\PRL{{Phys. Rev. Lett.}}
\def\PR{{Phys. Rep.}}
\def\PLB{{Phys. Lett. B}}
\def\AP{{Ann. Phys. (N.Y.)}}
\def\EPJA{{Eur. Phys. J. A}}
\def\NP{{Nucl. Phys.}}  
\def\RMP{{Rev. Mod. Phys.}}
\def\IJMPE{{Int. J. Mod. Phys. E}}
\def\AJ{{Astrophys. J.}}
\def\AJL{{Astrophys. J. Lett.}}
\def\AA{{Astron. Astrophys.}}
\def\ARAA{{Annu. Rev. Astron. Astrophys.}}
\def\MPLA{{Mod. Phys. Lett. A}}
\def\ARNPS{{Annu. Rev. Nucl. Part. Sci.}}
\def\LRR{{Living. Rev. Rel.}}
\def\CQG{{Classical Quantum Gravity}}
\def\RAS{{Mon. Not. R. Astron. Soc.}}

\title{}

\author{}

\affiliation{}

\begin{abstract}

\end{abstract} 

\keywords{}
\pacs{}

%-------------------------------------------------------------------------
\maketitle
%%%%%%%%%%%%%%%%%%%%%%%%%%%%%%%%%%%%%%%%%%%%%%%%%%%%%%%%%%%%%%%%%%%%%%%%%%  
 \section{INTRODUCTION}
\label{sec_intro}



\section{FORMALISM}
\label{formalism}
\subsection{Generalized Uncertainty Principle}

\subsection{Impact of Generalized Uncertainty Principle on Nuclear Matter}
\label{RMF}
Nuclear matter and finite nuclei can be describe by RMF models. The Lagrangian density of RMF models is defined as~\cite{SB2012}:
\begin{equation}
\label{eq:LagrangianTotal}
\mathcal{L} = \mathcal{L}_N + \mathcal{L}_M  + \mathcal{L}_{int},
\end{equation}
where the contribution  of free nucleons in finite nuclei is
\begin{equation}
\label{eq:LagrangianBarion}
\mathcal{L}_N = \sum_{N}\overline{\psi}_N\left(i\gamma_{\mu}\partial^{\mu} - M_N\right)\psi_N,
\end{equation}
with the sum taken over all nucleons $N$ in nuclei. Nuclear matter is thermodynamic limit of finite nuclei where  in this limit, $N \rightarrow \infty$, volume $\rightarrow \infty$ but the densities are finite. Therefore, in this limit, we have $\sum_{N} \rightarrow \int d^3k$. Note that the interactions between nucleons are mediated by the exchange of scalar-isoscalar ${\sigma}$, vector-isosaclar  ${\omega}$, and vector-isovector ${\rho}$, mesons, respectively. Furthermore, the corresponding mesons have self-interactions. The interaction Largrange density for finite nuclei  taken following form~\cite{HP2001}:
\begin{eqnarray}
\label{eq:LagrangianInteraksi}
\mathcal{L}_{int} &=& \sum_{N}g_{\sigma}\sigma\overline{\psi}_N\psi_N - \sum_{N}g_{\omega}V_{\mu}\overline{\psi}_N\gamma^{\mu}\psi_N \nonumber \\
&-&\sum_{N}g_{\rho}{\bold b}_{\mu}\cdot\overline{\psi}_{N}\gamma^{\mu} {\boldsymbol \tau}\psi_N - \frac{1}{3} b_2\sigma^3 - \frac{1}{4}b_3\sigma^4 \nonumber \\
&+& \frac{1}{4}c_3\left(V_{\mu}V^{\mu}\right)^2\nonumber \\&+& d_2\sigma \left(V_{\mu}V^{\mu}\right) +f_2\sigma \left({\mathbf b}^{\mu}\cdot {\mathbf b}_{\mu}\right)+ \frac{1}{2}d_3\sigma^2\left(V_{\mu}V^{\mu}\right).
\end{eqnarray}
For free mesons, the Lagrangian density is as follows
\begin{equation}
\label{eq:LagrangianMeson}
\mathcal{L}_M = \mathcal{L}_{\sigma} + \mathcal{L}_{\omega} + \mathcal{L}_{\rho}, \\
\end{equation}
where the explicit form of each term is 
\begin{eqnarray}
\mathcal{L}_{\sigma} &=& \frac{1}{2}\left(\partial_{\mu}\sigma\partial^{\mu}\sigma - m_{\sigma}^2\sigma\right) \label{eq:SigmaMeson}, \\
\mathcal{L}_{\omega} &=& -\frac{1}{2}\left(\frac{1}{2}\omega_{\mu\nu}\omega^{\mu\nu} - m_{\omega}^2V_{\mu}V^{\mu}\right) \label{eq:OmegaMeson}, \\
\mathcal{L}_{\rho} &=& -\frac{1}{2}\left(\frac{1}{2}{\boldsymbol \rho}_{\mu\nu}\cdot {\boldsymbol \rho}^{\mu\nu} - m_{\rho}^2{\mathbf b}_{\mu} \cdot {\mathbf b}^{\mu}\right) \label{eq:RhoMeson}.
\end{eqnarray}
Within the mean field approximation, $\sigma$, ${V}^{\mu}({V}_0, 0)$, and ${\mathbf b}^{\mu}({\mathbf b}_0, 0)$ are ${\sigma}$, ${\omega}$, and ${\rho}$ fields, respectively, and $\omega_{\mu\nu}$ and ${\boldsymbol \rho}_{\mu\nu}$ are the anti-symmetric tensor fields of ${\omega}$ and ${\rho}$ meson. Note that for the case NS matter, the $\beta$-stability condition should be satisfied. Therefore, the electrons and muons (leptons) should be exist in the NS matter. The contribution of non-interacting leptons to the total Lagrangian density is as follows
\begin{equation}
\label{eq:LagrangianLepton}
\mathcal{L}_L = \sum_{L}\overline{\psi}_L\left(i\gamma_{\mu}\partial_{\mu} - m_L\right)\psi_L.
\end{equation}

In the following we will discuss the impact of the phase space deformation due to GUP on nuclear matter and NS. However, due to the leptons contribution on EOS of a NS is not significant, to simplify the problem, we assume that the phase space of leptons do not deform due to GUP; therefore, the expression for the zero component of the vector (lepton number) density and the energy density and pressure derived from Eq.~(\ref{eq:LagrangianLepton}) take the standard expressions~\cite{Glendenning}. Using the RMF calculation procedure~\cite{Glendenning}, we obtained the modified nucleon number densities for matter due to phase space deformation caused by GUP as
\bea
  \label{NLnumberden}
  \rho^*_N=\frac{2}{{(2 \pi)}^3} \int_0^{k_f^N } \frac{d^3k}{\left(1+\beta k^2\right)^2},~~~N=p,n.
\eea  
Similarly, scalar number densities for protons and neutrons are expressed as follows:
\begin{equation}
  \label{NLscalar_numberden}
  \rho^*_{s~N}=\frac{2}{{(2 \pi)}^3}\int_0^{k_f^N}\frac{M^*_N}{\sqrt{\frac{1}{\beta}{\left(\tan^{-1} [\sqrt{\beta} k]\right)}^2+M_N^{*~2}}}\frac{d^3k}{\left(1+\beta  k^2\right)^2},
 \end{equation}  
 where $M^*_N=M_N+g_\sigma \sigma$. The total energy density can be calculated as follows:
 \begin{equation}
   \label{eden}
\epsilon=\sum_{N=n,p}\epsilon_N^*+g_\omega (\rho^*_p+\rho^*_n)+\frac{1}{2}g_\rho (\rho^*_p-\rho^*_n)+U+\sum_{L=e,\mu}\epsilon_L,
\end{equation}
where the meson contribution is as follows:
\begin{eqnarray}
U&=&\frac{1}{2}m_s^2 \sigma^2-\frac{1}{2}m_\omega^2 V_0^2+\frac{1}{2}m_\rho^2 b_0^2
\nonumber\\ &+& \frac{1}{3}b_2\sigma^3+\frac{1}{4}b_3 \sigma^4-\frac{1}{4}c_1 V_0^4\nonumber \\ &-& d_2\sigma V_0^2-f_2\sigma b_0^2- \frac{1}{2}d_3\sigma^2V_0^2,
\end{eqnarray}
and the nucleon contributions are as follows:
\begin{eqnarray}
  \label{nuceden}
  \epsilon_N^*= \frac{2}{ {(2 \pi)}^3} \int_0^{k_f^N }  \sqrt{\frac{1}{\beta}{\left(\tan^{-1} [\sqrt{\beta} k]\right)}^2+M_N^{*~2}}\frac{d^3k}{\left(1+\beta  k^2\right)^2}.
\end{eqnarray}
The explicit expressions for $P_{r}$ is
\begin{equation}
  \label{press}
P_{r}=\sum_{M=n,p}\frac{1}{3}P^{*~N}_{r}-U+\sum_{L=e,\mu}\frac{1}{3}P_L,
\end{equation}
with\cite{Corchero1998}
\begin{eqnarray}
  \label{nucpress}
  P_r^{*~N}= \frac{2}{{(2 \pi)}^3}\int_0^{k_f^N}\frac{k^2}{\sqrt{\frac{1}{\beta}{\left(\tan^{-1} [\sqrt{\beta} k]\right)}^2+M_N^{*~2}}}\frac{d^3k}{\left(1+\beta  k^2\right)^2}
\end{eqnarray}



\subsection{Neutron Stars Model}


\section{RESULTS AND DISCUSSIONS}
\label{RAD}


\section{CONCLUSIONS}
\label{sec_conclu}



%%%%%%%%%%%%%%%%%%%%%%%%%%%%%%%%%%%%%%%%%%%%%%%%%%%%%%%%%%%%%%%%%%%%%%%%%%
\section*{ACKNOWLEDGMENT}


%%%%%%%%%%%%%%%%%%%%%%%%%%%%%%%%%%%%%%%%%%%%%%%%%%%%%%%%%%%%%%%%%%%%%%%%%

\begin {thebibliography}{100}
\bibitem{SB2012}A. Sulaksono and B. K. Agrawal,
\Journal{\NPA}{895}{44}{2016}.
 \bibitem{HP2001}C. J. Horowitz and J. Piekarewicz,
\Journal{\PRL}{86}{5647}{2001}.
\bibitem{Glendenning} N. K. Glendenning,{\it Compact Stars, Nuclear Physics, Particle Physics, and General Relativity} (Springer, New York, 1997).

 \end{thebibliography}

\end{document}
